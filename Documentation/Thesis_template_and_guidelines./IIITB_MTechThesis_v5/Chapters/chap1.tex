% chap1.tex

\chapter{Introduction}\label{chap:intro}

Visualization has become an indispensable tool in many areas of science and engineering. In particular, the advances in visualization made over the past twenty years have turned visualization from a presentation tool to a discovery tool.

Volume visualization is a technique which enables physicians and scientists to gain insight into complex volumetric structures. Currently, the trend towards information acquisition using data sets from multiple modalities is increasing in order to facilitate better medical diagnosis. As different modalities frequently carry complementary information, our goal is to combine their strengths and generate focus+context  visualization.


\section{Motivation}



For volume models, the key advantage of using direct volume rendering is its potential to show the structure of the value distribution throughout the volume. The contribution of each volume sample to the final image is explicitly computed and included. The key challenge of direct volume rendering is to convey that value distribution clearly and accurately. In particular, showing each volume sample with full opacity and clarity is impossible if volume samples in the rear of the volume are not to be completely obscured.

Despite the proliferation of volume rendering software, the design of effective transfer functions is still a challenge. The growing popularity of GPU-based volume renderers has advocated the use of a more exploratory approach, where users can arrive at good trans-
fer functions via trial-and-error modification of opacity and color values. However, effective transfer functions are often the product
of time-consuming tweaking of opacity parameters until meeting a desired quality metric, often subjective. One possible explanation
for this ad hoc methodology is the lack of an objective measure to quantify the quality of transfer functions. 


In interpreting volume data for surgical planning or medical diagnosis, the information which can be visualized from a single modality, example, Computed Tomography (CT), may be insufficient. A number of factors influence this, including limited resolution, sensitivity to tissue properties, noise, etc. For this reason, radiologists often make use of additional modalities that provide complementary or supplementary information. In this way, radiologists are able to extract more clearly the structures of interest and the spatial relationships
among them. For example, CT provides the most detailed anatomical information from the human body, usually at high resolution. It helps depict high dense structures such as bone, as well as the shape of internal organs. On the other hand, the acquisition of metabolic activity must rely on a modality like Positron Emission Tomography (PET). In general, metabolic activity is important to detect cancer, since cancer tumors and other malignancies are usually located in regions with high rate of metabolic activity, such as regions with high blood flow. To obtain the best of the two modalities, recent visualization systems attempt at fusing both types of information in a single meaningful image.


The issue of visibility is not exclusive of medical data. Simulations of 3D phenomena often contain structures that evolve and are intertwined in 3D space with other less interesting structures. Therefore, visualization of internal flow becomes difficult. 

With hardware acceleration, volume rendering has become very attractive to many applications. To be more widely adopated, however, its usability remains to be enhanced. In particular, the task of classifying volume data before rendering as well as the task of manipulating potentially a large number of rendering and viewing parameters to achieve desired visualization are often time-consuming and tedious. Recent research results show some good progress on visualizing individual volume data, but multimodal volume rendering presents additional challenges, from the problems of superimposing dual modality data and highlighting objects of interest, to the desire to suppress occluding materials while maintaining the context and to enhance structural and spatial clarity of the objects.


\section{Background/Related Work}






\section{Objective}







\section{Non Photorealistic Rendering }

The emergence of non-photorealistic rendering (NPR) over the greater part of a decade has created an intriguing new field espousing expression, abstraction and stylisation in preference to the traditional computer graphics concerns for photorealism. By lifting the burden of realism, NPR is capable of engaging with users, providing compelling and unique experiences through devices such as abstraction and stylisation. Non-photorealistic rendering can be used to illustrate subtle spatial relationships that might not be visible with more realistic rendering techniques. 

Volume rendering has remained a prevalent tool in medical and scientific visualisation for over a decade. The ability to visualise complex real-world phenomena has found its way into practical applications including CT and MRI scans of the brain and imaging flow in fluid dynamics. The integration of volume rendering with non-photorealistic rendering(NPR) is an intuitive and natural progression given the communicative and expressive capabilities of NPR. 

Volume Non-photorealistic rendering acheives two complimentary goals, the communication of information using images and rendering images in interesting and novel visual styles which are free of the traditional computer graphics constraint of producing images which are “life-like”. Hence, Volume non photorealistic rendering techniques can be used to create visualizations of volume data that are more effective at conveying the structure within the volume.

[1] Volume Illustration: Non-Photorealistic Rendering of Volume Models
[2] State of the Art Non-Photorealistic Rendering (NPR) Techniques


\section{Focus and Context for Volume Visualization }

In the case of volume data (3D datasets), direct volume rendering [4] is one of the most used approaches for visualization. Medical applications are amongst the most popular ones, data acquired from a scanner (computerized tomography, magnetic resonance, etc) is fed to a volume rendering system, allowing physicians and radiologists to see internal structures and organs with much greater detail than with conventional methods.

However, in some cases there is too much data to be displayed at once on a computer display (or the display’s resolution may be insufficient for practical use). A simple and widely used solution is to apply a magnification factor to get closer to a specific region. But by doing so, it is equally easy to get lost in the dataset. This is generally called loss of context, because we are no longer able to visualize the entire dataset. When we zoom in, we are focusing on a certain feature that is of interest. In the field of Visualization this problem is called focus+context [22] and a number of successful solutions have emerged. The challenge is to find a way of looking at a high level of detail at this area of focus, without losing the overall context.

[22] Robert Spence. Information Visualization. ACM Press, 1st edition, 2001.
[4] Robert A. Drebin, Loren Carpenter, and Pat Hanrahan. Volume rendering. In Proceedings of the 15th annual conference on Computer graphics and interactive techniques, pages 65–74. ACM Press, 1988.

\section{Organization of thesis}
Thesis is be divided into two parts. First part deals with visibility histogram, which represents the visibility of the sample values from a given viewport. These visibility histogram provides a feedback mechanism for designing transfer function. Visibility histograms are view and opacity dependent. This method becomes an important aid for volume exploration.
Therefore, first part of thesis we deal with defining visibility driven transfer functions. 

Second part of deals with challenges posed by multimodal visualization to generate informative pictures from complementary data(we used CT and PET ). The visibility information is used to fuse multimodal datasets for generating focus+context visualization.
Using visibility calculations, tradeoff between visibility and spatial clarity is handled. 

\section{Chapters}
The thesis should contain an abstract, acknowledgments, chapters, and
bibliography. Appendices are optional. The chapters should include an
introducted, related work (or literature survey), background, hypothesis,
implementation, results, and conclusions.

chapter 1: Introduction
chapter 2: Background/Literature survey
chapter 3: Direct Volume rendering using raycasting
chapter 4: Visibility Histograms
chapter 5: Visibility guided multimodal volume visualization
chapter 6: Prerequisite Installations
chapter 7: Implementation
chapter 8: Results \& Future Work



